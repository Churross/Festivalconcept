\begin{song}[vals]{Granada}
\rhythmandkey[\nicefrac{3}{4}]{Vals}{G}
\begin{instrumental}{Intro}
\measure{E\_\_\_}\measure{F\_\_\_}\measure{E\_\_\_}\measure{E\_\_\_}\measure{F\_\_\_}\measure{Dm$\downarrow$}\measure*{E$\downarrow$E$\downarrow$E$\downarrow$}
\end{instrumental}

\begin{verse}{Couplet 1}
Gran\chord{Am\_\_\_}ada,\\
\chord{\_\_\_\_\_\_\_\_\_\_\_\_\_\_}tierra soñada por mi,\\
\chord{\_\_\_\_\_\_\_\_\_\_\_\_\_\_\_\_\_}mi cantar se vuelvo \chord{G\_\_\_\_\_\_\_\_\_\_\_\_\_\_}gitano cuando \chord{F\_\_\_\_}es para \chord{E\_\_\_}ti\\
Mi cantar hecho de fantasia,\\
mi cantar flor de melancolia,\\
que yo te vengo a dar.\\
\end{verse}

\begin{verse}{Couplet 2}
Gran\chord{C}ada,\\
tierra \chord{Am}\hspace{1em} ensangrent\chord{Em}ada en t\chord{F\#dim$\downarrow$}ardes de t\chord{G7}oros\\
muj\chord{G}er que cons\chord{G4-9}erva el embr\chord{G}ujo de l\chord{G5+}os o\chord{$\downarrow$}jos m\chord{C}oros \\
de s\chord{C}ueño reb\chord{Am}elde y git\chord{Em}ana cub\chord{F\#dim}ierta \chord{$\downarrow$}de fl\chord{Em}ores\\
y \chord{B7}beso tu boca de g\chord{Em}rana jugosa man\chord{B7}zana\\
que me habla de a\chord{Em}more\chord{G$\downarrow$}s.\\
\end{verse}

\begin{verse}{Couplet 3}
Gran\chord{C}ada,\\
ma\chord{Am}nola can\chord{Em}tada en \chord{F\#dim}coplas \chord{$\downarrow$}\hspace{0.5em} prec\chord{G7}iosas.\\
No t\chord{G}engo \chord{G4-9}otra cosa que d\chord{G}arte que un \chord{G5+}ramo \chord{$\downarrow$}\hspace{0.5em} de r\chord{C}osas\\
de rosas de s\chord{C7}uave fr\chord{F}agancia\\
que \chord{Fm}le dieron m\chord{C}arco a la \chord{Fm6}Virgen mor\chord{C}ena,  \chord{C$\downarrow$}\\
\end{verse}

\begin{verse}{Couplet 4a}
Granada,\\
tu tierra está llena de lindas mujeres,\\
de sangre y de sol.\\
\end{verse}

\begin{instrumental}{Intermezzo}
\measure{C}\measure{Am}\measure{Em}\measure{F\#dim$\downarrow$}\measure{G7}\measure*{}\measure{}\measure{}\measure{G}\measure{G4-9}\measure{G}\measure*{G5+ $\downarrow$}
\end{instrumental}

\begin{verse}{Couplet 4b}
de rosas de suave fragancia\\
que le dieron marco a la Virgen morena,\\
Granada,\\
tu tierra está llena de lindas mujeres,\\
\end{verse}
\clearpage

\end{song}

